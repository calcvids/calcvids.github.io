\documentclass[12pt]{report}

\usepackage{times} %Times New Roman?
\usepackage{mathtools} %More Math Tools
\usepackage{changepage} %Changing Margins and Indent
\usepackage{extramarks} %Required for headers/footers?
\usepackage{enumerate}
\usepackage{graphicx} %Insert pictures
\usepackage{empheq} %Emphasize equations
\usepackage[none]{hyphenat}
\usepackage[titles]{tocloft}
\usepackage{multicol}
\usepackage[margin=1in]{geometry} %1inch margins?
\usepackage{multirow}
\pagenumbering{gobble}
\usepackage{fancyref}
\usepackage[explicit]{titlesec}
\usepackage{setspace}
\linespread{1} %Line Spacing
\everymath{\displaystyle}
\usepackage{fancyhdr}
\usepackage{amsfonts}
\usepackage{amsthm}
\usepackage{amsmath}
\usepackage{tabularx}


\begin{document}
\newcommand{\less}{\textless}
\newcommand{\greater}{\textgreater}
\newcommand{\reals}{\mathbb{R}}
\newcommand{\integers}{\mathbb{Z}}
\newcommand{\rationals}{\mathbb{Q}}
\newcommand{\dsp}{\displaystyle}

\section*{Limit Definition of a Derivative}

\begin{enumerate}

\item Consider the function defined by $f(x) = x^2 - 1$. What is the value of $ \lim_{h \to 0}\frac{f(3+h)-f(3)}{(3+h)-3}$?

\begin{enumerate}

\item $0$

\item $6$

\item 8

\item $2x$

\item The limit does not exist.

\end{enumerate}

\item The expression $\lim_{h \to 0}\frac{(x+h)^3-\ln(x+h) - \left(x^3-\ln(x)\right)}{h}$ is the derivative of what function?

\begin{enumerate}

\item $f(x) = (x+h)^3-\ln(x+h)$

\item $f(x) = 3x^2 - \frac{1}{x}$

\item $f(x) = 3x^2 - \frac{1}{x}$

\item $f(x) = x^3 - \ln(x)$

\item $f(x) = \frac{(x+h)^3-\ln(x+h) - \left(x^3-\ln(x)\right)}{h}$

\end{enumerate}

\item What is the instantaneous rate at which the volume of an $8in^3$ cube grows as its side lengths increase from a single vertex on the left-most face?


\item If $f$ is a differentiable function and $a$ is a number, then $f^\prime(a)$ is given by which of the following expressions: \vspace{0.2cm}
\begin{enumerate}
\item [I.] $\dsp \lim_{h \to 0}\frac{f(a+h)-f(a)}{h}$ \vspace{0.2cm}
\item [II.] $\dsp \lim_{x \to a}\frac{f(x)-f(a)}{x-a}$ \vspace{0.2cm}
\item [III.] $\dsp \lim_{h \to 0}\frac{f(x+h)-f(x)}{x-h}$\\
\end{enumerate}
\begin{enumerate}
\item [a.] I only
\item [b.] II only
\item [c.] I and II only
\item [d.] I and III only
\item [e.] I, II, and III
\end{enumerate}

\item  The following expression represents the derivative of what function?\vspace{0.25cm}
\[
\lim_{\Delta x \to 0}\frac{2(x+\Delta x)^7-5(x+\Delta x)+8-\left(2x^7-5x+8\right)}{\Delta x}
\]\vspace{0.25cm}
\begin{enumerate}
\item [a.] $f(x) = 2(x+\Delta x)^7-5(x+\Delta x)+8$ \vspace{0.2cm}
\item [b.] $f(x) = 2x^7-5x+8$ \vspace{0.2cm}
\item [c.] $f(x) = 2(x+\Delta x)^7-5(x+\Delta x)+8-\left(2x^7-5x+8\right)$ \vspace{0.2cm}
\item [d.] $f(x) = 14x^6-5$ \vspace{0.2cm}
\item [e.] $\dsp f(x) = \frac{2(x+\Delta x)^7-5(x+\Delta x)+8-\left(2x^7-5x+8\right)}{\Delta x}$
\end{enumerate}

\item$\dsp \lim_{h \to 0} \frac{(2+h)^4-2^4}{h}=$ \vspace{0.25cm}
\begin{enumerate}
\item [a.] 0 \vspace{0.2cm}
\item [b.] 16 \vspace{0.2cm}
\item [c.] 1 \vspace{0.2cm}
\item [d.] 32 \vspace{0.2cm}
\item [e.] The limit does not exist
\end{enumerate}

\item The differentiable function $g$ is increasing over the interval $(x_0, \, x_0+1)$. If $x_0 < c < x_0+1$, what can you conclude about $\lim_{x \to c}\frac{g(x)-g(c)}{x-c}$? Explain your response. 

\end{enumerate}



\end{document}