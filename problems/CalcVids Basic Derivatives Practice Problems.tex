\documentclass[12pt]{report}

\usepackage{times} %Times New Roman?
\usepackage{mathtools} %More Math Tools
\usepackage{changepage} %Changing Margins and Indent
\usepackage{extramarks} %Required for headers/footers?
\usepackage{enumerate}
\usepackage{graphicx} %Insert pictures
\usepackage{empheq} %Emphasize equations
\usepackage[none]{hyphenat}
\usepackage[titles]{tocloft}
\usepackage{multicol}
\usepackage[margin=1in]{geometry} %1inch margins?
\usepackage{multirow}
\pagenumbering{gobble}
\usepackage{fancyref}
\usepackage[explicit]{titlesec}
\usepackage{setspace}
\linespread{1} %Line Spacing
\everymath{\displaystyle}
\usepackage{fancyhdr}
\usepackage{amsfonts}
\usepackage{amsthm}
\usepackage{amsmath}
\usepackage{tabularx}


\begin{document}
\newcommand{\less}{\textless}
\newcommand{\greater}{\textgreater}
\newcommand{\reals}{\mathbb{R}}
\newcommand{\integers}{\mathbb{Z}}
\newcommand{\rationals}{\mathbb{Q}}
\newcommand{\dsp}{\displaystyle}


\section*{Basic Derivative Rules}

\begin{enumerate}

\item Give two explanations for the identity $\frac{d}{d\theta}(\sin(\theta))=-\frac{d}{d\theta}(\sin(\theta+\pi))$. 

\item A cube grows in volume as the length of its edges grow (expanding from a single corner). How quickly does the volume of an $8$ $in^3$ cube grow with respect to its growing side lengths? After you determine the growth rate, make a geometric argument supporting your claim. 

\item Consider a right triangle has base length $x$ and an area of $2$. Suppose that the base length begins to change but the area remains fixed at $2$. If the base length is $5$ and begins to decrease, what is the rate of change of the height?

\item In chemistry, pH is a scale used to measure the acidity of a solution. The pH of a solution is defined by the equation
\[
\text{pH} = -\log_{10}(x)
\] 
where $x$ represents the concentration of hydrogen ions. Compute the rate of change of pH with respect to hydrogen ion concentration when the pH is 2. (\textit{Note that} pH \textit{is a single quantity, not the product of} p \textit{and} H.)

\item Without using the chain rule, why is the derivative of $f(x)=\ln(x^5)=\frac{5}{x}$


\end{enumerate}



\end{document}