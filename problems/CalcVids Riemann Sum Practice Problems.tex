\documentclass[12pt]{report}
%\usepackage{amsmath,amsthm} %More Math Imputs
%\usepackage{amssymb} %More Math Symbols
%\usepackage{amsfonts} %More Math Fonts
\usepackage{times} %Times New Roman?
\usepackage{mathtools} %More Math Tools
\usepackage{changepage} %Changing Margins and Indent
%\usepackage{fancyhdr} %Fancy Header for making headers
\usepackage{extramarks} %Required for headers/footers?
\usepackage{enumerate} %Enumerate is half-assed....
\usepackage{graphicx} %Insert pictures
\usepackage{empheq} %Emphasize equations
\usepackage[none]{hyphenat}
\usepackage[titles]{tocloft}
\usepackage{multicol}
\usepackage[margin=1in]{geometry} %1inch margins?
\usepackage{multirow}
\pagenumbering{gobble}
\usepackage{fancyref}
\usepackage[explicit]{titlesec}
\usepackage{setspace}
%Margins
%\topmargin=-0.45in
%\evensidemargin=0in
%\oddsidemargin=0in
%\textwidth=6.5in %text area
%\textheight=9in %text area
%\headsep=.25in %separation between header and stuff
\linespread{1} %Line Spacing
\newcommand{\R}{\mathbb{R}}
\everymath{\displaystyle}
\usepackage{fancyhdr}
%\lhead{Zack Reed\\}
%\chead{RUME 2017 Proposal Tex}

%\lfoot{}
%\cfoot{\thepage}
%\rfoot{}
%\renewcommand\headrulewidth{0.4pt} %thickness of header line
%\renewcommand\footrulewidth{0.4pt} %thickness of footer line
\usepackage{amsfonts}
\usepackage{amsthm}
\usepackage{amsmath}
\usepackage{tabularx}
\setlength\parindent{.25in} % no indentation in the whole document
% Theorem, Definition, etc.... MACROS
\newtheorem{axiom}{Axiom}[subsection]
\newtheorem{definition}{Definition}[subsection]
\newtheorem{statement}{Statement}[subsection]
\newtheorem{comment}{Comment}[subsection]
\newtheorem{convention}{Convention}[subsection]
\newtheorem{proposition}{Proposition}[subsection]
\newtheorem{lemma}{Lemma}[subsection]
\newtheorem{theorem}{Theorem}[subsection]
\newtheorem{corollary}{Corollary}[subsection]
\newtheorem{properties}{Properties}[subsection]
\newtheorem{formula}{Formula}[subsection]
\newtheorem{example}{Example}[subsection]
%\newtheorem*{axiom}{Axiom}
\DeclareMathOperator{\Log}{Log}
\DeclareMathOperator{\Res}{Res}
\DeclareMathOperator{\Arg}{Arg}
\DeclareMathOperator{\Ind}{Ind}
\DeclareMathOperator{\Real}{Re}
\DeclareMathOperator{\Imag}{Im}
%\begin{definition}\textbf{}\\
%\end{definition}
%\begin{theorem}\textbf{}\\
%\end{theorem}
\newenvironment{myquote1}%
  {\list{}{\leftmargin=0.5in\rightmargin=.25in\linespread{1}}\item[]}%
  {\endlist}%Sets margins for blockquotes
 \newenvironment{myquote2}%
  {\list{}{\leftmargin=0.25in\rightmargin=0.25in}\item[]}%
  {\endlist}%Sets margins for blockquotes
  \usepackage{pdfpages}
\usepackage{graphicx}
%\setcounter{secnumdepth}{4}

\let\LaTeXStandardTableOfContents\tableofcontents

\renewcommand{\tableofcontents}{%
\begingroup%
\renewcommand{\bfseries}{\relax}%
\LaTeXStandardTableOfContents%
\endgroup%
}%
\usepackage[english]{babel}
\addto\captionsenglish{% Replace "english" with the language you use
  \renewcommand{\contentsname}%
    {\center{\normalsize TABLE OF CONTENTS}   \\  \hfill \normalsize\renewcommand{\bfseries}{\relax}\underline{Page}\vspace{-12pt}}%
    }

\setcounter{secnumdepth}{5}

\AtEndDocument{\thispagestyle{fancy}}
\AtEndDocument{ \fancyhead[R]{\thepage}}
\AtEndDocument{  \renewcommand{\headrulewidth}{0pt}}

\begin{document}
\newcommand{\epzero}{\forall\epsilon\textgreater0}
\newcommand{\epone}{\textless\epsilon}
\newcommand{\la}{\left|}
\newcommand{\ra}{\right|}
\newcommand{\lno}{\left\|}
\newcommand{\rno}{\right\|}
\newcommand{\xinf}{x\rightarrow\infty}
\newcommand{\less}{\textless}
\newcommand{\greater}{\textgreater}
\newcommand{\bff}{\begin{definition}}
\newcommand{\eff}{\end{definition}}
\newcommand{\reals}{\mathbb{R}}
\newcommand{\exN}{\exists N\in\mathbb{N}}
\newcommand{\tvect}[3]{%
\newcommand{\grad}{\nabla}
\newcommand{\weakarrow}{\hookrightarrow}
  \ensuremath{\Bigl(\negthinspace\begin{smallmatrix}#1\\#2\\#3\end{smallmatrix}\Bigr)}}
  \newcommand{\lnorm}{\left\|}
\newcommand{\rnorm}{\right\|}
\newcommand{\integers}{\mathbb{Z}}
\newcommand{\rationals}{\mathbb{Q}}
\newcommand{\Q}{\mathbb{Q}}
\newcommand{\F}{\mathbb{F}}
\newcommand{\jq}{\emph{Jerry}:}
\newcommand{\cq}{\emph{Christina}:}
\newcommand{\iq}{\emph{Interviewer}:}
\newcommand{\Lq}{\emph{Laura}:}
\newcommand{\N}{{\Bbb N}}
\newcommand{\Z}{{\Bbb Z}}
\newcommand{\dsp}{\displaystyle}
\newcommand{\blank}{\makebox[.5in]{\hrulefill}}
\newcommand{\be}{\begin{enumerate}}
\newcommand{\ee}{\end{enumerate}}
\newcommand{\limx}[1]{\lim_{x\rightarrow#1}}
\newcommand{\s}{\hskip6pt}

\section*{Riemann Sums}

\begin{enumerate}

%\item Let $f$ be a positive, strictly increasing function on $[2, 4]$. Which of the following approximations of $\int_2^4 f(x)\,dx$ is the largest?
%\begin{enumerate}

%\item $R_4$

%\item $M_4$

%\item $L_4$

%\item They are all equal

%\item There is not enough information provided to determine which approximation is largest.

%\end{enumerate}

\item Kacey decides to go for a run before school. She starts her run from home. The function $y=v(t)$ expresses the relationship between Kacey's velocity (in meters per minute) as she runs and the number of minutes $t$ elapsed since she started running. What quantity does the sum 

$$v(1)\cdot \frac{1}{2}+v\left(3/2\right)\cdot \frac{1}{2}+v\left(5/2\right)\cdot \frac{1}{2}+v\left(3\right)\cdot \frac{1}{2}+v\left(7/2\right)\cdot \frac{1}{2}$$

approximate?

\begin{enumerate}

\item The average rate of change of Kacey's velocity over the interval of time from $ t = 1$ to $t = \frac{7}{2}$.

\item The change in Kacey's distance away from home over the interval of time from $t = 1$ to $t = 4$.

\item Kacey's acceleration over the interval of time from $t=1$ to $t=4$

\item Kacey's distance away from home after having run for 3.5 minutes.

\item Kacey's instantaneous velocity 3.5 minutes after having left home.
\end{enumerate}

The function $y=g(t)$ represents the relationship between the rate of change in the value of investment stocks (in dollars per month) and the number of months $t$ elapsed since the stocks were purchased. Which of the following sums approximates the change in the value of the stocks over the interval of time from 4 to 7 months after the stocks were purchased?

\begin{enumerate}

\item $\sum_{k=4}^7g(k)$

\item $\sum_{k=4}^7g(t)\cdot\Delta t$

\item $\sum_{k=1}^6g(4+.5k)\cdot .5$

\item $\sum_{k=0}^3g(4+k)\cdot \Delta t$

\item $\sum_{k=0}^3g(4+k)$

\end{enumerate}

% PROBLEM 4(b)
\item[4(b)] (\textit{5 points}) The following image illustrates a Riemann sum using $N$ terms:\\
\begin{center}
\includegraphics[width=5.5in]{Graph_2_2016.png}
\end{center}
Write each item on the right in the blank next to the corresponding expression on the left. Items B, C, and $A_3$ refer to the labeled graphical quantities above. \textit{\textbf{Each expression will be used exactly once.}} 
\begin{center}
\begin{tabular}{c c c c c}
	\textbf{\underline{Expression}} & & & & \textbf{\underline{Item}}\\\\
	$\Delta x$ & \underline{\hspace{2.25in}} & & & B\\\\
	$f(x_3)\Delta x$ & \underline{\hspace{2.25in}} & & & C\\\\
	$f(x_3)$ & \underline{\hspace{2.25in}} & & & $A_3$\\\\
	$\dsp \sum_{i=1}^N f(x_i)\Delta x$ & \underline{\hspace{2.25in}} & & & $\dsp \int_a^b f(x)dx$\\\\
	$\dsp \lim_{N \to \infty} \sum_{i=1}^N f(x_i)\Delta x$ & \underline{\hspace{2.25in}} & & & $A_1+A_2+A_3+\cdots+A_{N-1}+A_N$\\
\end{tabular}
\end{center}

\item Explain why $\sum_{k=1}^N f\left(3+\frac{(k-1)}{N}\right)\cdot\frac{1}{N}$ is an under approximation of the area bounded by the graph of $f$ and the $x$-axis when $f$ is increasing on the interval $[3, 4]$.


\item Let $f(x)$ represent the linear density (in g/m) of a 20 meter long wire, where $x$ is the distance in meters from one end. The definite integral $\int_0^{20} f(x)\,dx$ is approximated by the right endpoint approximation with $N$ terms:

$$\sum_{i=1}^N f(i\Delta x)\cdot \Delta x$$

Note that this Riemann sum is based on a uniform partition.

Explain what the following expressions represent {\bf in the context of the wire} and provide its units of measurement. 

\begin{enumerate}

\item $\Delta x$

\item $i\Delta x$

\item $f(i\Delta x)$

\item $f(i\Delta x)\cdot \Delta x$

\item $\sum_{k=1}^Nf(i\Delta t)\cdot \Delta t$

\item $\lim_{n\rightarrow \infty}\sum_{i=1}^Nf(i\Delta t)\cdot \Delta t$

\end{enumerate}

\item Let $f(t)$ represent the horizontal velocity (in ft/s) of a golf ball $t$ seconds after it was struck and lands $b$ seconds later. The definite integral $\int_0^b f(t)\,dt$ is approximated by the right endpoint approximation with $N$ terms:

$$\sum_{i=1}^N f(i\Delta t)\cdot \Delta t$$

Note that this Riemann sum is based on a uniform partition.

Explain what the following expressions represent {\bf in the context of the golf ball} and provide its units of measurement. 

\begin{enumerate}

\item $\Delta t$

\item $i\Delta t$

\item $f(i\Delta t)$

\item $f(i\Delta t)\cdot \Delta t$

\item $\sum_{k=1}^Nf(i\Delta t)\cdot \Delta t$

\item $\lim_{N\rightarrow\infty}\sum_{i=1}^Nf(i\Delta t)\cdot \Delta t$

\end{enumerate}

\pagebreak

\item The following image illustrates a Riemann sum using $N$ terms:

\begin{center}
	\includegraphics[width = 5in]{Riemann_Sum_S20.png}
\end{center}

$A_{3}$= the area of the third rectangle from the left

$B =$ the vertical height of the third rectangle

$C =$ the horizontal width of the third rectangle

$D =$ the total shaded area of the LaTeX: NN rectangles

Match each graphical quantity with the corresponding expression on the left from the definition of the integral

$\int_a^bf(x)\,dx=\lim_{N\to\infty}\sum_{i=1}^Nf(x_i)\Delta x$

Each item will be used exactly once.

\begin{enumerate}

\item $\Delta x$

\item $f(x_3)$

\item $f(x_3)\Delta x$

\item $\sum_{i=1}^Nf(x_i)\Delta x$

\end{enumerate}

\item The expression $\sum_{k=1}^{10}\left(1+\frac{3k}{10}\right)^3\cdot\frac{3}{10}$  is a Riemann sum approximation for which of the following?

\begin{enumerate}
\item $\frac{3}{10}\int_1^{10}x^3\ dx$
\item $\int_1^{10}(1+x)^3\ dx$
\item $\frac{3}{10}\int_1^{10}\left(1+\frac{3x}{10}\right)^3\ dx$
\item $\int_1^4x^3\ dx$
\item $\frac{3}{10}\int_1^4x^3\ dx$
\end{enumerate}

\item The expression $\frac{1}{10}\left(\left(\frac{1}{10}\right)^2+\left(\frac{2}{10}\right)^2+\left(\frac{3}{10}\right)^2+\ldots +\left(\frac{20}{10}\right)^2\right)$ is a Riemann sum approximation for which of the following expressions?
\begin{enumerate}

\item $\int_0^2x^2\ dx$
\item $\int_0^2\left(\frac{x}{10}\right)^2\ dx$
\item $\frac{1}{10}\int_0^2\left(\frac{x}{10}\right)^2\ dx$
\item $\int_0^{20}x^2\ dx$
\item $\frac{1}{10}\int_0^2x^2\ dx$
\end{enumerate}


\item The expression $\sum_{k=1}^{6}\sin(2+.5k)\cdot.5$  is a Riemann sum approximation for which of the following?

\begin{enumerate}
\item $.5\int_1^{6}\sin(2+.5x) dx$
\item $\int_2^{5}\sin(x)\ dx$
\item $.5\int_1^{6}\sin(2+x)\ dx$
\item $\int_1^6\sin(x)\ dx$
\item $.5\int_2^5\sin(x)\ dx$
\end{enumerate}

\pagebreak

\item The expression $.04\left(\cos(1.04)+\cos(1.08)+\cos(1.12)+\ldots +\cos(3)\right)$ is a Riemann sum approximation for which of the following expression?

\begin{enumerate}

\item $\int_1^3\cos(x)\ dx$

\item $.04\int_{25}^{50}\cos(x)\ dx$

\item $\int_1^3\cos(1+.04x)\ dx$

\item $.04\int_0^1\cos(1+.04x)\ dx$

\item $\int_{1}^{50}\cos(x) dx$

\end{enumerate}

\end{enumerate}


\end{document}