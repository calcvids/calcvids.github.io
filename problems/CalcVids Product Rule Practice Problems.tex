\documentclass[12pt]{report}

\usepackage{times} %Times New Roman?
\usepackage{mathtools} %More Math Tools
\usepackage{changepage} %Changing Margins and Indent
\usepackage{extramarks} %Required for headers/footers?
\usepackage{enumerate}
\usepackage{graphicx} %Insert pictures
\usepackage{empheq} %Emphasize equations
\usepackage[none]{hyphenat}
\usepackage[titles]{tocloft}
\usepackage{multicol}
\usepackage[margin=1in]{geometry} %1inch margins?
\usepackage{multirow}
\pagenumbering{gobble}
\usepackage{fancyref}
\usepackage[explicit]{titlesec}
\usepackage{setspace}
\linespread{1} %Line Spacing
\everymath{\displaystyle}
\usepackage{fancyhdr}
\usepackage{amsfonts}
\usepackage{amsthm}
\usepackage{amsmath}
\usepackage{tabularx}


\begin{document}
\newcommand{\less}{\textless}
\newcommand{\greater}{\textgreater}
\newcommand{\reals}{\mathbb{R}}
\newcommand{\integers}{\mathbb{Z}}
\newcommand{\rationals}{\mathbb{Q}}
\newcommand{\dsp}{\displaystyle}


\section*{The Product Rule}

\begin{enumerate}

\item Use the product rule to show that $\frac{d}{dy}e^{3x}=3e^{3y}$

\item Suppose a rectangular prism has edge-lengths $f(t)$, $g(t)$, and $h(t)$. What is the rate of change in the volume of the rectangular prism with respect to time? Make a geometric argument to support the rate of change you find. 

\item Why does the area of a rectangle with side lengths $\sin(x)$ and $x$ grow more slowly near $x=0$ than near $x=2\pi$? Why does the area continue to grow more quickly with each multiple of $2\pi$ (i.e. $x=4\pi$, $x=6\pi$, etc.) even though the area is always $0$ at multiples of $2\pi$?

\item Use the product rule to find the derivative of $\frac{3\cos(x)+1}{4x^5}$

\end{enumerate}



\end{document}